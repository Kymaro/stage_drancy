\chapter{Configuration de \textit{Microsoft Azure}}\\

\section{Mise en place pour 1 station}\\

Nous allons abandonner notre \textit{RaspberryPi} pendant quelques instant pour se focaliser sur la création d'une instance dans notre \textit{Cloud} pour que l'on puisse envoyer les données.\\

Pour réaliser cette partie, munissez vous de vos identifiant \textit{Microsoft Azure}.\\

\subsection{Explication des fonctionnalités proposé par \textit{Microsoft}}
Avant de commencer, nous allons détailler un peu plus précisément le cheminement que vont avoir nos données. Comme souvent une image vaut mieux qu'un long discours, je vous laisse observer : 
%dessin ici photoshop Hub -> Time Serie Insight

%description de l'image a faire ici

Je vais vous expliciter deux méthodes différentes de gestion de \textit{Azure} pour vous montrer le champs des possible. En effet, nous pouvons utiliser le portail qui nous est proposé par \textit{Microsoft} ou bien utiliser ce qui s'appelle \textit{L'Azure CLI} pour \textit{Azure Command Line Interface}.

\subsection{Création d'un Hub d'évènements sur \textit{Azure} avec la \textit{CLI}}\\

\subsubsection{Installation de l'Interface en Lignes de Commande de \textit{Azure}}\\

Pour cette partie il n'est donc pas nécessaire de se rendre sur le portail. Mais avant toute chose, vous devez installer l'interface en ligne de commandes \textit{d'Azure}.\\

\begin{enumerate}
	\item Installer \href{https://nodejs.org/download/release/latest/}{\textit{Node.js}} en choisissant l'extension \textit{.msi} pour x64 ou x32 selon si votre système est en 64bit ou 32bit.
	\item Lancer un terminal \textit{git bash}.
	\item Tapez les deux commandes suivantes pour vérifier que \textit{Node.js}  et \textit{NPM} (Node Package Manager) a bien été installé. \textit{NPM} est l'utilitaire qui va nous permettre d'installer \textit{Azure}
	\begin{lstlisting}[style=MyBashStyle]
	node -v
	npm -v
	\end{lstlisting}\\
	Vous devriez avoir ce résultat : %photo ici
	\item Ensuite entrer la commande :
	\begin{lstlisting}[style=MyBashStyle]
	npm install -g azure-cli
	\end{lstlisting}\\
\end{enumerate}\\
Et voilà, Azure est installé, vous devriez pouvoir le lancer en tapant \textit{azure} dans votre terminal \textit{git bash}. Voici ce que vous devriez obtenir : %photo ici

\subsubsection{Création d'un groupe de ressources et d'un Hub d'évènement}\\

Maintenant, il va falloir créer nos instance qui vont récupérer les valeurs envoyée par notre station.

\begin{enumerate}
	\item Connecter vous avec votre compte \textit{Azure} :
	\begin{lstlisting}[style=MyBashStyle]
	azure login
	\end{lstlisting}\\
	Aller alors sur la page : \href{https://aka.ms/devicelog}{https://aka.ms/devicelog} pour entrer le code qui vous sera spécifié sur votre terminal. Entrer ensuite vos identifiants. Vous devriez alors être connecté. %photo ici
	\item Taper la commande :
	\begin{lstlisting}[style=MyBashStyle]
	azure group list
	\end{lstlisting}\\
Vous verrez alors la liste des groupes de ressources déjà existant. Nous allons alors créer un groupe de ressources par la commande : 
	\textbf{NOTE : POUR LE DOSSIER, LE GROUPE A POUR NOM \textit{IoT}. VOUS POUVEZ RENSEIGNER LE NOM QUE VOUS SOUHAITEZ}\\
	\begin{lstlisting}[style=MyBashStyle]
	azure group create IoT westeurope
	\end{lstlisting}\\
westeurope détermine le serveur où le groupe sera créé. %photo ici
\\
Si vous refaites la commande de l'étape N°2, vous devriez voir apparaitre votre nouveau groupe fraichement créé sur la liste.
	\item Maintenant, vous allez devoir télécharger le code de la station, comme nous l'avons fait pour la \textit{RaspberryPi} dans le chapitre 2. En effet, il y a un fichier qui sert de patron pour les paramètres à appliquer pour la commande qui suivra. Ainsi, entrer les commandes : 
	\begin{lstlisting}[style=MyBashStyle]
	cd
	git clone XXXXXXXXXXXXXXXXXXXXXXXXXXXXXXXXXXXXXXXXXXX
	\end{lstlisting}\\
	\item Maintenant, nous pouvons faire la commande :
	\begin{lstlisting}[style=MyBashStyle]
	azure group deployment create -g IoT -n deployment1 -f ~/stage_drancy/IoT.json
	\end{lstlisting}\\ %PATH a changer

Vous allez être inviter à entrer différentes informations. Les informations que vous allez rentrer seront indispensable car devront être ajouté au code de la station pour pouvoir envoyer les données. Les valeurs sont des chaines de caractère, celle mise en photo sont arbitraires pour le dossier, veuillez entrer une valeur différente.
	\begin{enumerate}
		\item namespaceName : Il est conseillé de rajouter NS à la fin pour vous y retrouver ensuite. %photo ici
		\item eventHubName : Il s'agit de l'instance à proprement parlé qui va recevoir les données. vous pouvez par exemple entrer une valeur du type "Station".
		\item consumerGroupName : Pour celle valeur, je vous conseille d'entrer la même que pour namespaceName mais de remplacer NS par GN.
	\end{enumerate}\\ 
Le déploiement peut prendre un peu de temps. Si cela ne fonctionne pas, retenter en entrant des nom différents.
Une fois le déploiement terminé. Plusieurs valeurs sont à noté et sauvegarder précieusement.

\textbf{Notez d'une part la valeur des trois champs que vous avez indiquer !.
De plus, il vous faut noter la valeur de \textit{SharedAccessKeyName} et \textit{SharedAccessKey}}

\end{enumerate}

Nous en avons terminé avec la création de l'instance qui va récupérer les données envoyé par la station. Nous allons maintenant nous occuper de l'instance qui va traiter les données reçu.

\subsection{Création d'un portail \textit{Time Series Insights}}

\textit{Microsoft} à pensé à nous puisque il a crée un module qui s'occupe seul de traiter les données d'une source pour ensuite créer des courbes et ainsi voir le suivi dans le temps. Sans cela, le processus aurait été nettement plus long puisqu'il aurait fallut d'une part stocké ces valeurs pour ensuite les traiter et devoir créer une interface qui puisse afficher les valeurs.

Nous allons pour cette partie nous rendre sur le \href{https://portal.azure.com}{\textit{Portail Azure}}. Nous allons dans un premier temps jeter un coup d'oeil sur le groupe que nous avons créé dans la partie d'avant. %photo ici 
\\

Cliquer à gauche sur \textit{Groupes de ressources}. Vous devriez voir apparaitre un Hub D'évènement qui porte le nom de votre \textit{namespaceName} de la partie précédente. Maintenant si vous cliquez sur ce Hub d'évènement, en bas de la vue d'ensemble devrait apparaitre un Hub d'évènement dont le nom est celui que vous avez entré pour le champs eventHubName précédemment. Une fois n'est pas coutume, continuons notre jeux des poupées russes et cliquer dessus. Une fois encore, vous devriez voir apparaitre en bas deux groupe de consommateur. Le premier, \textit{Default} est.... celui utilisé par défaut si jamais vous ne voulez pas créer de groupe de consommateur. Mais c'est toujours mieux d'avoir un dont on connait le nom par soucis de visibilité. Le second est celui que vous avez créé tout à l'heure.\\

Maintenant que vous êtes rassuré sur les lignes de commande de la partie précédente nous allons pouvoir créer notre instance de traitement des données.\\
\begin{enumerate} %photo a chaque étape ici.
	\item Cliquer sur le \textit{+} en haut à gauche de votre portail.
	\item Dans la barre de recherche du menu qui vient de s'ouvrir, entrer \textit{"Time Series"}
	\item Choisissez alors \textit{"Time Series Insights (aperçu)} %ajout photo ici

	
	




