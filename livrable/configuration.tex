\chapter{Configuration de la station}

\section{Création et installation de la distribution \textit{Raspbian}.}

Pour une \textit{RaspberryPi}, son disque dur est nul autre qu'une carte Micro SD. C'est donc sur ce support que vous allez faire l'installation.

Pour réaliser cette étape, vous aurez besoin de :
\begin{itemize}
	\item d'une \textit{RaspberryPi}
	\item d'une carte micro SD de minimum 8Go
	\item du logiciel \href{etcher.io}{Etcher}
	\item de \href{https://sourceforge.net/projects/dexterindustriesraspbianflavor/}{\textit{Raspbian}}. En réalité il s'agit d'une version modifiée par la société Dexter Industries qui est spécialisé dans l'utilisation de \textit{RaspberryPi} pour la robotique.
	\item d'un adaptateur pour relier la micro SD à votre ordinateur.
\end{itemize}\\
Vous pouvez désormais commencer l'installation de l'OS sur notre \textit{RaspberryPi}

\begin{enumerate}
	\item Connecter la micro SD sur votre ordinateur.\\
	\item Lancer le logiciel \textit{Etcher}\\
	\begin{figure}[H]
	\begin{center}
		\makebox[\textwidth]{\includegraphics[width=.6\paperwidth]{images/etcher1.jpg}}
	\end{center}
		\caption{ \textit{Logiciel Etcher au lancement}}
	\end{figure}\\
	\item Cliquer sur \textit{"Select image"} à gauche puis sélectionner le fichier \textit{.zip} récupéré depuis le site \textit{SourceForce} dans les pré-requis.\\
	\item Vérifier que le périphérique qui est renseigné au centre est bien la micro SD. Dans le cas contraire cliquer sur \textit{"Change"}.\\
\newpage
	\item Si les deux étapes précédentes sont OK, cliquer sur \textit{"Flash"}.\\
	\begin{figure}[H]
	\begin{center}
		\makebox[\textwidth]{\includegraphics[width=.6\paperwidth]{images/etcher2.jpg}}
	\end{center}
		\caption{ \textit{Etcher prêt à flasher}}
	\end{figure}\\
\end{enumerate}
Et voilà, l'opération peut prendre une dizaine de minutes. C'était simple non ?

\section{Première mise en route de la \textit{RaspberryPi}.}

\begin{enumerate}

	\item Maintenant que vous avez \textit{Raspbian}, vous allez pouvoir démarrer votre nouvel ordinateur.
Pour cela, il vous suffit d'insérer la microSD au dos de la \textit{RaspberryPi}.\\
\begin{figure}[H]
\begin{center}
	\makebox[\textwidth]{\includegraphics[width=.6\paperwidth]{images/microsd.jpg}}
\end{center}
	\caption{ \textit{La \textit{RaspberryPi} de dos}}
\end{figure}\\

	\item Avant de l'alimenter, nous allons distinguer deux cas.. Si vous avez la possibilité, de façon extérieur, d'accéder à l'adresse IP de votre \textit{RaspberryPi} alors vous pouvez sauter l'étape N°3. Si cela n'est pas possible, brancher un écran et un clavier & souris.\\
	Vous pouvez maintenant l'alimenter en utilisant le port micro USB qui est à côté du port HDMI.\\

\newpage
	
	\item Si vous suivez cette étape, vous devriez voir apparaître des lignes de commandes qui défilent. Quelques instants plus tard, vous arrivez sur l'environnement de bureau de votre \textit{RaspberryPi}. Sur ce bureau, ouvrez un terminal en cliquant sur l'écran en haut à gauche : 
	
\begin{figure}[H]
\begin{center}
	\makebox[\textwidth]{\includegraphics[width=.6\paperwidth]{images/terminal.jpg}}
\end{center}
	\caption{ \textit{Barre de tâche de Raspbian}}
\end{figure}\\

Une fois le terminal ouvert, entrer la commande suivante :
\begin{lstlisting}[style=MyBashStyle]
	sudo ifconfig
\end{lstlisting}\\

un mot de passe devrait vous être demandé, par défaut, le mot de passe est "robots1234".
Vous devriez avoir un résultat de ce type :\\

\begin{figure}[H]
\begin{center}
	\makebox[\textwidth]{\includegraphics[width=.6\paperwidth]{images/ip_mac.jpg}}
\end{center}
	\caption{ \textit{Résultat de la commande sudo ifconfig}}
\end{figure}\\

Vous pouvez ainsi récupérer l'adresse IP et adresse MAC si nécessaire pour gérer votre réseau. Pour l'adresse IP, selon l'architecture de votre réseau, il s'agit d'une adresse privé ou public. Dans le premier cas, vous serez obligé d'être dans le même réseau pour y accéder, dans le second pas de problème quelque soit l'endroit où vous vous trouver.
Maintenant que vous avez récupérer l'adresse IP, vous pouvez débrancher tous les périphériques de votre \textit{RaspberryPi} et laisser uniquement l'alimentation et le câble ethernet.

	\item Très bien, désormais que vous avez l'adresse IP à disposition, vous pouvez installer le nécessaire pour utiliser nos capteurs. Vous pourriez très bien continuer cette installation directement sur la \textit{RaspberryPi} mais si vous n'avez jamais fait ce qui va suivre, cela vous fera un bon entrainement.\\

\begin{enumerate} 
	\item Télécharger et installer le logiciel \href{https://git-for-windows.github.io/}{Git Bash}. Ce logiciel permet l'utilisation d'un terminal plus avancé et compatible avec le langage système \textit{Bash}.
	\item Lancer \textit{Git Bash}
	\item taper la commande en remplaçant "xxx.xxx.xxx.xxx" par l'adresse IP de la \textit{RaspberryPi}\\
	\begin{lstlisting}[style=MyBashStyle]
	ssh pi@xxx.xxx.xxx.xxx
	\end{lstlisting}\\
la première fois, il vous sera demander si vous voulez réellement vous connecter sur le périphérique, taper alors "yes" puis sur la touche \textit{Entrée}
	\item Le mot de passe est "robots1234". Si tout c'est bien passé, vous devriez avoir cet aperçu :\\
	\begin{figure}[H]
	\begin{center}
		\makebox[\textwidth]{\includegraphics[width=.6\paperwidth]{images/ssh.jpg}}
	\end{center}
		\caption{ \textit{connexion à la RaspberryPi en SSH}}
	\end{figure}\\
	
	\item vous naviguez maintenant dans la \textit{RaspberryPi}. Dans un premier temps, il va falloir changer le mot de passe car celui ci est un mot de passe par défaut. Enter la commande :\\
	\begin{lstlisting}[style=MyBashStyle]
	sudo raspi-config
	\end{lstlisting}\\
	un menu sur fond bleu devrait apparaitre :\\
	
\begin{figure}[H]
\begin{center}
	\makebox[\textwidth]{\includegraphics[width=.6\paperwidth]{images/raspiconfig.jpg}}
\end{center}
	\caption{ \textit{Menu Raspi-config}}
\end{figure}\\

Premièrement, choisissez la première option \textit{Expand Filesystem}. Cela va faire en sorte que \textit{Raspbian} occupe toute la micro SD. Une fois terminé, choisissez la seconde option. Nous allons changer le mot de passe par défaut "robots1234". Il va vous être demandé de saisir 2 fois un nouveau mot de passe. Pas de panique, lorsque l'on tape un mot de passe sous linux, aucun caractère apparait !
	
	\item Quitter le menu en choisissant l'option \textit{"Finish"}. Il vous sera demandé de redémarrer la \textit{RaspberryPi}. Après quelques temps, refaite l'étape \textit{c & d} mais avec le nouveau mot de passe qui a été renseigné.
	\item Maintenant, vous pouvez installer GrovePi pour pouvoir utiliser le \textit{Shield}. Entrer alors les deux commandes suivantes :\\
	\begin{lstlisting}[style=MyBashStyle]
	sudo curl https://raw.githubusercontent.com/DexterInd
	/Raspbian_For_Robots/master/upd_script/fetch_grovepi.sh | bash
	 
	sudo reboot
	\end{lstlisting}\\
	Votre \textit{RaspberryPi} va redémarrer.
	\item Connecter vous à nouveau en "ssh" comme pour l'étape 4 mais cette fois ci avec votre nouveau mot de passe.
	\item Réaliser alors cette suite de commande une à une. Appuyer sur la touche \textit{Entrée} lorsque l'on vous demande de continuer :
	\begin{lstlisting}[style=MyBashStyle]
	cd
	sudo git clone https://github.com/DexterInd/GrovePi
	cd GrovePi/Script
	sudo chmod +x install.sh
	sudo ./install.sh
	\end{lstlisting}\\
	
	\item Vous devriez à la fin obtenir un écran similaire à la photo précédente.\\ 
	Effectuer alors la commande :
	\begin{lstlisting}[style=MyBashStyle]
	sudo shutdown now
	\end{lstlisting}\\
	Elle va alors s'arrêter. Vous pouvez alors la débrancher après quelques instant.
	\end{enumerate}\\
	
		\item Vous pouvez désormais ajouter le \textit{Shield} sur la \textit{RaspberryPi} Comme ci-dessous \textbf{ATTENTION AU BROCHES UTILISÉES, BRANCHER LE SHIELD DANS LA MÊME CONFIGURATION QUE SUR LA PHOTO !}\\
		
	\begin{figure}[H]
	\begin{center}
		\makebox[\textwidth]{\includegraphics[width=.6\paperwidth]{images/branchement.jpg}}
	\end{center}
		\caption{ \textit{La RaspberryPi avec le Shield branché dessus}}
	\end{figure}\\
	
		\item Brancher à nouveau la \textit{RaspberryPi} au secteur. Vous devriez avoir une LED verte qui s'allume sur votre \textit{Shield}
		\item vous allez tester s'il a bien été reconnue, pour cela, connecter vous en "ssh" sur votre \textit{RaspberryPi} (vous devriez savoir le faire maintenant !)
		\item lancer la commande :
		\begin{lstlisting}[style=MyBashStyle]
	sudo i2cdetect -y 1
		\end{lstlisting}\\
	
vous devriez obtenir ce résultat, avec le 04 en première ligne.\\

\begin{figure}[H]
\begin{center}
	\makebox[\textwidth]{\includegraphics[width=.6\paperwidth]{images/i2cdetect.jpg}}
\end{center}
	\caption{ \textit{Test de détection du Shield}}
\end{figure}\\

\end{enumerate}\\

Voilà, la première mise en route de la \textit{RaspberryPi} est terminé. Maintenant il faut s'occuper du code pour la station final.

\newpage
\section{Installation, configuration et test du code de la station avec ses capteurs.}\\

\subsection{Installation des capteurs.}\\

Votre \textit{RaspberryPi} est configurée, ainsi que son \textit{Shield}. Il faut installer les différents capteurs. Regardons de plus près les différents connecteurs.\\

	\begin{figure}[H]
	\begin{center}
		\makebox[\textwidth]{\includegraphics[width=.6\paperwidth]{images/grovepi_connecteur.jpg}}
	\end{center}
		\caption{ \textit{vu de dessus du shield}}
	\end{figure}\\
	
Comme vous pouvez le voir, il y a écrit un numéro d'identification du connecteur sur le \textit{Shield}. Vous allez donc brancher sur des ports particulier en adéquation avec le code qui sera téléchargé ultérieurement.\\

Procéder aux branchement branchements suivants :
\begin{enumerate}
	\item Le capteur de température et d'humidité (DHT11) sur le port D7.
	\item Le capteur de luminosité sur le port A1.
	\item L'encodeur rotatif sur le port A2.
	\item L'écran LCD sur un des ports I2C.
\end{enumerate}
Vous devriez alors obtenir un résultat similaire à celui là :
	\begin{figure}[H]
	\begin{center}
		\makebox[\textwidth]{\includegraphics[width=.35\paperwidth]{images/branchement_final.jpg}}
	\end{center}
		\caption{ \textit{Branchement des compostant sur le shield}}
	\end{figure}\\

\subsection{Configuration de la \textit{RaspberryPi}.}\\

Vous allez maintenant télécharger le code pour activer le station. Pour cela, brancher votre \textit{RaspberryPi} au secteur et au réseau si jamais vous aviez enlevé le câble ethernet. Connectez vous en \textit{ssh} via \textit{git bash} et exécutez les deux commandes :\\
		\begin{lstlisting}[style=MyBashStyle]
	cd
	git clone https://github.com/Kymaro/stage_drancy
		\end{lstlisting}\\
%mettre la commande git clone du repo final

\\
Cette commande va télécharger le code nécessaire au fonctionnement de la station. Vous modifierez ce code dans le troisième chapitre pour envoyer les données sur le \textit{Cloud}.
\\

Tapez la commande :\\

\begin{lstlisting}[style=MyBashStyle]
	ls | grep stage_drancy
\end{lstlisting}\\

\begin{figure}[H]
\begin{center}
	\makebox[\textwidth]{\includegraphics[width=.6\paperwidth]{images/lsgrep.jpg}}
\end{center}
	\caption{ \textit{Résultat de la commande ls avec grep}}
\end{figure}\\ 

Si vous avez bien ce résultat, cela signifie que le code à bien été téléchargé et qu'il est présent dans le dossier \textit{stage_drancy}. Si cela n'est pas le cas, réessayer la commande précédente.\\

Puis il faut faire en sorte que le script soit lancé au démarrage de la \textit{RaspberryPi}. Pour cela, il faut modifier un ficher qui gère les commandes système qui sont lancé lorsque l'OS démarre.\\

Exécuter la commande : \\

\begin{lstlisting}[style=MyBashStyle]
	sudo nano /etc/rc.local
\end{lstlisting}\\

Un fichier va alors s'ouvrir dans votre terminal, vous ne pouvez pas utiliser la souris dans cet éditeur de texte. Descendez alors avec les flèches directionnelles jusqu'à l'avant dernière ligne, la dernière ligne étant normalement \textit{exit 0}. Allez à la fin de la ligne et appuyez alors sur \textit{Entrée} pour ajouter une nouvelle ligne. 
\\
Tapez alors sur cette nouvelle ligne :\\

\begin{lstlisting}[style=MyBashStyle]
	sudo python /home/pi/stage_drancy/rpi/test_complet.py &
\end{lstlisting}\\

L' esperluette permet d'exécuter le script en tâche de fond, il est important de ne pas l'oublier sinon votre \textit{RaspberryPi} ne sera pas capable de faire autre chose que de lancer votre script.

Pour quitter et enregistrer, taper sur la combinaison de touche une à une :\\

\begin{lstlisting}[style=MyBashStyle]
	Ctrl + X
	Y
	Entree
\end{lstlisting}\\

Vous êtes désormais prêt à tester la station.

\subsection{Test de la station}\\

Vous allez tout d'abord essayer la station en lançant le script manuellement pour vérifier que cela fonctionne, puis nous redémarrerons la \textit{RaspberryPi} pour voir si la modification de l'étape précédente est fonctionnelle.\\

Tapez la commande suivante pour lancer le script python.\\

NOTE : Lorsque vous indiquez le chemin d'un fichier en \textit{Bash}, vous avez votre meilleur ami qui est là pour vous aider et qui s'appelle l'auto-complétion. En effet, lorsque vous commencer à écrire le nom d'un fichier, par exemple \textit{station_cloud}, il vous suffit d'écrire \textit{sta} puis d'appuyer sur la touche \textit{Tabulation} pour voir la suite se compléter. Si cela ne se complète pas, c'est que vous avez un autre fichier / dossier qui commence par \textit{sta}, vous devez alors écrire un caractère supplémentaire pour retenter votre chance ;)\\

\begin{lstlisting}[style=MyBashStyle]
	cd
	sudo python stage_drancy/rpi/test_complet.py &
\end{lstlisting}\\

Cela ne vous rappelle rien ? Il s'agit ni plus ni moins de la ligne qui a été ajoutée dans le fichier \textit{/etc/rc.local} pour démarrer automatiquement le script. (il manque /home/pi/ puisque nous sommes dans ce dossier par défaut).

Vous devriez voir l'écran LCD de la station qui s'allume avec le message "Bienvenue dans l'IoT Hub" pendant quelques seconde puis un des menus qui affiche les données des capteurs. Si cela reste sur le premier message, tourner l'encodeur rotatif pour afficher les menus.

Si vous voyez bien tous les menus avec une valeur, on y est, la station fonctionne. Vérifiez maintenant qu'elle démarre bien en même temps que la \textit{RaspberryPi}.
Mettez vous dans le scénario où elle a été coupée électriquement. 
Taper la commande :\\ %shutdown now.

\begin{lstlisting}[style=MyBashStyle]
	sudo shutdown now
\end{lstlisting}\\

Patienter quelques instants, débrancher puis rebrancher là. Vous devriez alors voir l'écran s'allumer comme pour le test précédent au bout d'un certain temps. %image.

Et voilà,vous êtes au terme de ce chapitre, vous avez désormais une station fonctionnelle localement. Je vous invite alors à suivre le chapitre trois pour configurer \textit{Microsoft Azure} d'une part puis pour modifier le code de tel sorte qu'il envoie les messages sur \textit{le Cloud Azure}.



	
	
